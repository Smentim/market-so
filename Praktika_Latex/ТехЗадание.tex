\section{Техническое задание}
\subsection{Основание для разработки}

Основанием для разработки является задание на выпускную квалификационную работу бакалавра "<Маркетплейс для реализации цифровых предметов внутри игры \textquotedbl Stay Out\textquotedbl">.

\subsection{Цель и назначение разработки}

Основной задачей выпускной квалификационной работы является разработка и внедрение web-сайта для упрощения торговли внутриигровыми ценностями в «Stay Out».


В процессе разработки web-сайта для упрощения торговли внутриигровыми ценностями в «Stay Out» можно выделить следующие задачи:
\begin{itemize}
\item реализация системы аутентификации и безопасности;
\item разработка функционала поиска и фильтрации товаров;
\item внедрение системы обратной связи;
\item разработка системы управления контентом;
\item внедрение механизмов модерации.
\end{itemize}

\subsection{Требования пользователя к интерфейсу web-сайта}

Сайт должен включать в себя:
\begin{itemize}
    \item поиск предметов;
    \item авторизацию;
    \item регистрацию;
    \item выбор сервера;
    \item доступ к профилю.
\end{itemize}

Композиция шаблона сайта представлена на рисунке ~\ref{fig:image}.
\begin{figure}[ht]
	\centering
	\includegraphics[width=1\linewidth]{images/image}
	\caption{Композиция шаблона сайта}
	\label{fig:image}
\end{figure}
%\vspace{-\figureaboveskip} % двойной отступ не нужен (можно использовать, если раздел заканчивается картинкой)

\subsection{Функциональные требования к программной системе}

Для разрабатываемого сайта была реализована модель, которая обеспечивает наглядное представление вариантов использования сайта.

Она помогает в физической разработке и детальном анализе взаимосвязей объектов. При построении диаграммы вариантов использования применяется унифицированный язык визуального моделирования UML.

Диаграмма вариантов описывает функциональное назначение разрабатываемой системы. То есть это то, что система будет непосредственно делать в процессе своего функционирования. Она является исходным концептуальным представлением системы в процессе ее проектирования и разработки. Проектируемая система представляется в виде ряда прецедентов, предоставляемых системой актерам или сущностям, которые взаимодействуют с системой. Актером или действующим лицом является сущность, взаимодействующая с системой извне (например, человек, техническое устройство). Прецедент служит для описания набора действий, которые система предоставляет актеру.

На основании анализа предметной области в разрабатываемой программно-информационной системе должны быть реализованы следующие функции:
\begin{enumerate}
\item Регистрация нового пользователя.
\item Вход пользователя в систему.
\item Просмотр списка выставленных предметов.
\item Поиск предметов.
\item Просмотр подробной информации о товаре.
\item Выставление предмета на продажу/покупку.
\item Управление личным профилем.
\item Отправка сообщений другим пользователям.
\item Получение сообщений от других пользователей.
\end{enumerate}

На рисунке ~\ref{fig:-} в виде диаграммы прецедентов представлены функциональные требования к системе для пользователей.

\begin{figure}[th]
	\centering
	\includegraphics[width=1\linewidth]{"images/Диаграмма прецедентов"}
	\caption{Диаграмма прецедентов}
	\label{fig:-}
\end{figure}

\subsubsection{Вариант использования «Выбор сервера»}

Заинтересованные лица и их требования: пользователь желает начать производить какие-либо действия на сайте, связанные с торговлей.
Предусловие: открыта главная страница сайта.
Постусловие: открывается нужная страница в соответствие с выбранным сервером.
Основной успешный сценарий:
\begin{enumerate}
	\item Пользователь нажимает на кнопку выбора сервера на главной странице сайта.
	\item На сайте открывается модальное окно с доступными серверами.
	\item Пользователь выбирает нужный ему сервер.
	\item Приложение открывает страницу, соответствующую выбранному серверу
\end{enumerate}

\subsubsection{Вариант использования «Смена сервера»}

Заинтересованные лица и их требования: пользователь играет сразу на нескольких серверах и хочет выполнить какие-либо действия, связанные с торговлей, на другом сервере.
Предусловие: открыта страница какого-либо сервера.
Постусловие: открывается нужная страница в соответствие с выбранным сервером.
Основной успешный сценарий:
\begin{enumerate}
	\item Пользователь нажимает на кнопку «Сменить сервер».
	\item На сайте открывается модальное окно с другими серверами.
	\item Пользователь выбирает нужный ему сервер.
	\item Приложение открывает страницу, соответствующую выбранному серверу
\end{enumerate}

\subsubsection{Вариант использования «Поиск»}

Заинтересованные лица и их требования: пользователь желает произвести поиск товаров, имеющих в названии одно или несколько ключевых слов.
Предусловие: открыта страница какого-либо сервера.
Постусловие: лента перезагружается, при наличии товаров, соответствующих поисковому запросу в базе данных, будет отображен их список.
Если товары отсутствуют, отображается пустой список.
Основной успешный сценарий:
\begin{enumerate}
	\item Пользователь переходит в режим поиска путем нажатия на элемент интерфейса (значок с изображением лупы) и ввода поискового запроса.
	\item Приложение передает поисковой запрос на сервер.
	\item Сервер формирует запрос в базу данных и передает результат в приложение.
	\item Приложение получает информацию от сервера и производит модификацию списка отображаемых товаров.
\end{enumerate}

\subsubsection{Вариант использования «Просмотр информации о товаре»}

Заинтересованные лица и их требования: пользователь желает получить больше информации о конкретном товаре.
Предусловие: открыта страница какого-либо сервера.
Постусловие: появляется модальное окно с полной информацией о товаре.
Основной успешный сценарий:
\begin{enumerate}
	\item Пользователь нажимает на значок с изображением нужного товара.
	\item Приложение получает уникальный id предмета и отправляет запрос на сервер.
	\item Сервер формирует запрос в базу данных и передает данные в приложение.
	\item Приложение получает информацию от сервера и заполняет модальное окно в соответствие с полученной информацией.
\end{enumerate}

\subsubsection{Вариант использования «Покупка товара»}

Заинтересованные лица и их требования: пользователь увидел интересующий его товар и захотел его приобрести.
Предусловие: открыта страница какого-либо сервер.
Постусловие: появляется уникальное сообщение, которое нужно скопировать и вставить в игре.
Основной успешный сценарий:
\begin{enumerate}
	\item Пользователь нажимает на кнопку «Купить».
	\item Приложение генерирует уникальное сообщение в соответствие с названием товаров и игровым никнеймом пользователя.
\end{enumerate}

\subsubsection{Вариант использования «Регистрация»}

Заинтересованные лица и их требования: пользователь хочет создать свой аккаунт на сайте.
Предусловие: пользователь еще не авторизовался на сайте.
Постусловие: регистрация произведена, появляется значек с изображение профиля.
Основной успешный сценарий:
\begin{enumerate}
	\item Пользователь нажимает на кнопку «Регистрация».
	\item Открывается экран регистрации пользователя.
	\item Пользователь вводит почту, логин, пароль и повторный пароль.
	\item Приложение отправляет запрос на сервер.
	\item Сервер формирует запрос в базу данных на создание аккаунта.
	\item Если аккаунта с таким же логином или почтой не существует, создается новый аккаунт.
	\item Сервер передает информацию в приложение о результате создания аккаунта.
	\item Приложение открывает окно пользователю с результатами регистрации.
\end{enumerate}

\subsubsection{Вариант использования «Авторизация»}

Заинтересованные лица и их требования: пользователь хочет войти в свой аккаунт на сайте.
Предусловие: пользователь еще не авторизовался на сайте.
Постусловие: авторизация произведена, появляется значок с изображение профиля.
Основной успешный сценарий:
\begin{enumerate}
	\item Пользователь нажимает на кнопку «Авторизация».
	\item Открывается экран регистрации пользователя.
	\item Пользователь вводит почту, логин, пароль и повторный пароль.
	\item Приложение отправляет запрос на сервер.
	\item Сервер формирует запрос в базу данных на создание аккаунта.
	\item Если аккаунта с таким же логином или почтой не существует, создается новый аккаунт.
	\item Сервер передает информацию в приложение о результате создания аккаунта.
	\item Приложение открывает окно пользователю с результатами регистрации.
\end{enumerate}

\subsubsection{Вариант использования «Выставление товара»}

Заинтересованные лица и их требования: пользователь хочет продать какую-либо вещь.
Предусловие: пользователь выбрал сервер и авторизовался.
Постусловие: предмет добавляется в ленту товаров:
\begin{enumerate}
	\item Пользователь нажимает на элемент интерфейса (значок с изображением плюса).
	\item Приложение открывает модальное окно с полями ввода.
	\item Пользователь вводит название предмета полностью или его часть.
	\item Приложение выводит все совпадения.
	\item Пользователь выбирает нужный ему предмет.
	\item Приложение отправляет запрос на сервер.
	\item Сервер формирует запрос в базу данных на получения все характеристик предмета.
	\item Сервер передает данные в приложение.
	\item Приложение заполняет необходимые данные в соответствие с данными.
	\item Пользователь вводит недостающие данные и выставляет предмет на продажу.
	\item Приложение отправляет запрос на сервер.
	\item Сервер вносит данные в базу данных.
	\item Сервер передает информацию о результате приложению.
	\item Приложение оповещает пользователя об успехе выставления предмета на продажу
\end{enumerate}

\subsection{Требования к оформлению документации}

Разработка программной документации и программного изделия должна производиться согласно ГОСТ 19.102-77 и ГОСТ 34.601-90. Единая система программной документации.
